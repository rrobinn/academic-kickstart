+++
title = "Restricted, Repetitive, and Reciprocal Social Behavior in Toddlers Born Small for Gestation Duration."
date = "2018-09-01"

# Authors. Comma separated list, e.g. `["Bob Smith", "David Jones"]`.
authors = ["Robin Sifre", " Carolyn Lasch", "Angela Fenoglio", "Michael Georgieff", "Jason Wolff", "Jed Elison"]
# Publication type.
# Legend:
# 0 = Uncategorized
# 1 = Conference proceedings
# 2 = Journal
# 3 = Work in progress
# 4 = Technical report
# 5 = Book
# 6 = Book chapter
publication_types = ["2"]

# Publication name and optional abbreviated version.
publication = "*Journal of Pediatrics*"
publication_short = "*J. Peds.*"

# Abstract and optional shortened version.
abstract = "OBJECTIVE:
To characterize restricted and repetitive behaviors (RRBs) and reciprocal social behaviors (RSBs) in a large sample of toddlers who represent a range of birth weights and gestation durations.

STUDY DESIGN:
A battery of questionnaires characterizing demographic information and measuring RRBs and RSBs were completed by parents of toddlers between the ages of 17-26 months (n = 1589 total; n = 98 preterm). The association between birth weight and/or gestation duration and the primary outcome measures (RRBs and RSBs as ascertained through the Repetitive Behavior Scale for Early Childhood and the Video-Referenced Rating of Reciprocal Social Behavior) were tested by using hierarchical multivariate multiple regression.

RESULTS:
Toddlers born preterm and full term did not differ on RRBs or RSBs. However, there were significant associations between birth weight percentile for gestation duration (BPGD) and RRBs (β = -2.1, P = .03), above and beyond the effects of age, sex, and vocabulary production. Similarly, there was a significant association between BPGD and RSBs (β = -1.8, P = .02), above and beyond the effects of age, sex, and vocabulary production.

CONCLUSIONS:
These findings demonstrate that BPGD better predicted putative antecedents of adverse psychological outcomes-specifically, RRBs and RSBs-than gestation duration alone. These findings provide insight to the link between preterm birth and suboptimal behavioral/psychological outcomes and suggest that high birth weight, which may reflect a more optimal intrauterine environment, may serve as a protective factor irrespective of gestation duration."
# Featured image thumbnail (optional)
image_preview = ""

# Is this a selected publication? (true/false)
selected = false

# Projects (optional).
#   Associate this publication with one or more of your projects.
#   Simply enter the filename (excluding '.md') of your project file in `content/project/`.
#projects = [""]

# Links (optional)# Custom links (optional).
# url_preprint = "https://dx.doi.org/10.17605/OSF.IO/JZA9R"
url_pdf = "https://www.sciencedirect.com/science/article/pii/S0022347618306504?via%3Dihub"
#url_code = "https://osf.io/yq52s/"
#url_dataset = "https://osf.io/yq52s/"
#url_project = ""
#url_slides = ""
#url_video = ""
#url_poster = ""
#url_source = ""


#   Uncomment line below to enable. For multiple links, use the form `[{...}, {...}, {...}]`.
#url_custom = [{name = "Preregistration", url = "https://osf.io/yq52s/"}, {name = "DOI", url = "https://#dx.doi.org/10.1177/1359105317723452"}]


# Does the content use math formatting?
math = true

# Does the content use source code highlighting?
highlight = true

# Featured image
# Place your image in the `static/img/` folder and reference its filename below, e.g. `image = "example.jpg"`.
[header]
image = ""
caption = ""

+++


